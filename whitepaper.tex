\def \projectname{STAKEPOOL\xspace}

\def \company{POS Mining CO}
\def \companyref{N/A}
\def \companytype{OTHER SMALL BUSINESS}
\def \team{N/A}
\def \restrictions{Proprietary Information}
\def \title{\projectname}
\def \biline{Proof Of Stake Mining Cryptocurrency}
\def \author{Your Name}
\def \email{stakepool@stakepool.co}
\def \phone{http://stakepool.co}
\def \address{Ontario, CA}

%%
% for each of the following items, uncommenting the fields will make them show up in the appropriate places
%% 

\def \placeofperformance{\address}
%\def \classification{For Official Use Only}   % only define if this is classified
%\def \exportcontrol{} % uncomment if this document is export controlled
%\def \proposalcontrol{} % uncomment if this is a proposal with restricted release for .gov purposes

%% the following are only needed for proposals.  
\def \baa{Whitepaper}
\def \techarea{Technical Area 1 (Your mom's tech area)}
\def \doctitle{Volume I (Technical and Management Proposal)}
\def \cost{\$1,000,000}

\def \duns{111111111}
\def \cagecode{222222}
\def \tin{33-3333333}
\def \awardtype{Cost Plus Fixed Fee (CPFF)}
\def \pop{January 1, 2000 - December 31, 2020 (3 days)}
\def \submitdate{January 1, 2020}

\input{includes/template.tex}

\begin{document}

\def\angle{0}
\def\radius{3}
\def\cyclelist{{"orange","blue","red","green"}}
\newcount\cyclecount \cyclecount=-1
\newcount\ind \ind=-1


\whitepapercover
%\docinfo

% mark pages that have restricted (aka: proprietary information with the "restricted" page style.
%\pagestyle{restricted}


\begin{abstract}
	STAKEPOOL is an ERC-20 Ethereum token representing the right of staking power on the StakePool.co network. StakePool.co uses Proof-Of-Stake(POS) mining along with masternodes to get the highest return on investments. The token will be ailable for purchase at the ICO crowdsale for a period of 60 days. No more than 50,00,000 tokens will be released during round 1. Round 2 will see another 100,000,000 tokens released as more coins move to proof-of-stake, including Ethereum. Round 2 won’t be launched until Ethereum is within 60-90 days of going POS. The remaining tokens will be held back for further tokens switching to POS/masternodes. Round 3 will only launch if needed at a much later date for future masternode coins. If no round 3, remaining tokens will be burned.
\end{abstract}

\section{The StakePool project}

The  StakePool  project  is  a   U.S.  company  (POS  Mining  Co.) based  in  Ontario,  CA  that  stakes  Proof-Of-Stake  coins  and  runs  masternodes  for income  purposes.  Our  green  facility  includes  solar  power  providing  25\%  of  our  electrical needs.  Each  POOL  token  represents  a   percentage  of  profits  and  tokens  generated  from POS  mining  and  will  be  paid  out  monthly  as  Ethereum  back  to  the  wallet  of  all  token holders.  As  with  most  coins,  we  expect  POOL  tokens  to  be  listed  on  several  market exchanges  for  trading  to  allow  participants  to  sell  or  buy  more  POOL  tokens.

\textbf{\emph{The ``one-sentence philosophy'' of proof of stake is thus not ``security comes from burning energy'', but rather ``security comes from putting up economic value-at-loss''.}}  \cite{vbuterinposphi}

 \subsection{Our vision}
  \subsubsection{History}
The founders of StakePool have been POS/Masternode mining for several clients for 2 years now. Starting with DASH, Blackcoin, Diamond, NEM and Reddcoin and later OkCoin, Stratis, and Decred, they began pooling coins together from both friends, family and others they met online to stake the largest rewards. As more funds came in, they began expanding to mine more and more coins. Unlike POW mining, the investment is solely in the coins. Electricity costs are minimal as is the upfront hardware costs. SECURITY: Unlike POW mining, POSi mining, the investment is solely in the coins. Electricity costs are minimal as is the upfront hardware costs.

  \subsubsection{Security}
Unlike POW mining, POS mining has it’s own security challenges. With POW mining, all rewards can be stored offline in hardware or paper wallets to protect against theft. With POS mining, all coins being staked must remain in unlocked wallets. StakePool has utilized various security measures, both physical and online to prevent both cyber and site location intrusions. StakePOOL uses enterprise-grade DDoS mitigation.

  \subsubsection{Transparency}
Like the underlying cryptocurrencies, we will be as open and transparent as possible while keeping security a top priority. By November 1, 2017, all wallet addresses will be posted online. This includes both Proof-Of-Stake and masternodes.

 \subsection{Technical side}
  \subsubsection{Why POS}
In 2015, the amount of electricity to mine a single bitcoin block would power 1.6 US homes per day. In 2016, it was 2.5 homes per bitcoin block. In a recent research paper, bitcoin transactions are expected to consume as much electricity as Denmark by 2020. As the planet switches to this new financial system, it is at odds with the green agenda.

Ethereum developers are worried about this problem and have set forth a greener consensus with the pending release of Casper which will move Ethereum away from Proof -Of-Work and onto the greener Proof-Of-Stake. Ethereum will join several other coins already using Proof-Of-Stake and have been successfully since 2012.

Not only is POS a fairer system, it is also several thousand times more cost effective. Token holder’s stake tokens in their wallets allowing them to mature, and leaving the wallets open. Investors who ’stake’ their tokens in such a way can earn rewards – a bit like earning interest on their holdings. POS is fairer for all and several thousand times more cost effective than POW. It doesn’t require any mining equipment so it isn’t subject to spiraling running costs and mining centralisation.

\textbf{\emph{However, there is one SHA256 alternative that is already here, and that essentially does away with the computational waste of proof of work entirely: proof of stake. Rather than requiring the prover to perform a certain amount of computational work, a proof of stake system requires the prover to show ownership of a certain amount of money.}} \cite{vbuterin2013}

  \subsubsection{POS Mining}
Proof of Stake (PoS) concept states that a person can mine or validate block transactions according to how many coins he or she holds in their wallet. This means that the more altcoin owned by a miner, the more mining power he or she has. The process of stake pool mining is keeping the altcoins out of cold storage on in an open wallet on the network. The staking wallets form a peer-to-peer network of wallets. Transactions can then be confirmed and verified and a reward is given. These rewards are what gets paid out to our POS token holders each month.

\subsubsection{Masternodes}
Look at a masternode as a special server that is maintained at all times. Masternodes are trustless and decentralized, similar to how bitcoin nodes operate. There is a major difference, though, as Dash masternodes take care of the anonymization part of the Darksend protocol. users can opt to send transactions anonymously by using this feature directly from their Dash wallet

Every masternode on the network provides this anonymization service, ensuring there is no centralized party to attack or take down. Moreover, masternodes ensure all transactions are validated in near real-time, making them quite efficient. Unlike bitcoin nodes, however, owners of a Dash masternode will receive a financial compensation for providing these invaluable services

\newpage

\section{Token Release}
The POOL token will be released as a pre-ico for early funders and adopters. Funds will be used to officially launch and fund coins for both masternodes and POOL coins. All funds raised at the pre-ico will be used for staking coins. There will be a bonus for the pre-ico phase of the crowdsale. A 10\% bonus will be applied to all token holders following the completion of the crowdsale.   

The official crowdsale for phase 1 won’t start until after the first payment is made to pre-ico token holders in October. Following this, phase 1 will launch and last 2 months. ETH collected during phase 1 will be again used solely for the purpose of new coins for stacking. There will be a 10\% holdback on ETH that won’t be used for staking from this phase and will be used for having some liquidity if something were to arise or a new opportunity arose for a masternode or staking. 

Phase 2 will be announced at a later date and will be used primarily for the accumulation of Ethereum for staking when it switches to Casper.

\begin{figure}[h]
\centering
\caption{Token distribution}
\begin{tikzpicture}[nodes = {font=\sffamily}]
	\foreach \percent/\name in {
		25/ICO  Round 1,
		50/ICO  Round 2,
		12.5/ICO  Round 3 (or burned),
		12.5/Founders (2-years lock)
	} {
		  \ifx\percent\empty\else               % If \percent is empty, do nothing
		  \global\advance\cyclecount by 1     % Advance cyclecount
		  \global\advance\ind by 1            % Advance list index
		  \ifnum3<\cyclecount                 % If cyclecount is larger than list
		  \global\cyclecount=0              %   reset cyclecount and
		  \global\ind=0                     %   reset list index
		  \fi
		  \pgfmathparse{\cyclelist[\the\ind]} % Get color from cycle list
		  \edef\color{\pgfmathresult}         %   and store as \color
		    % Draw angle and set labels
		  \draw[fill={\color!50},draw={\color}] (0,0) -- (\angle:\radius)
		  arc (\angle:\angle+\percent*3.6:\radius) -- cycle;
		  \node at (\angle+0.5*\percent*3.6:0.7*\radius) {\percent\,\%};
		  \node[pin=\angle+0.5*\percent*3.6:\name]
		  at (\angle+0.5*\percent*3.6:\radius) {};
		  \pgfmathparse{\angle+\percent*3.6}  % Advance angle
		  \xdef\angle{\pgfmathresult}         %   and store in \angle
		  \fi
	  };
\end{tikzpicture}
\end{figure}

\newpage

\section{Payouts}
The last week of the month, announcements will go out to alert everyone to get their POOL tokens to an Ethereum wallet and not an exchange. Following the last day of the month, a report will be posted to our Google docs site and announcement made. This spreadsheet will contain all token holders on record by the cutoff date and the amount of ETH each will be receiving to their address. On or around the 10th of the month, payments will be made in Ethereum to ethereum addresses on record at cutoff date. The splits will be 65/25/10. 65\% of the funds raised will be paid out in Ethereum to token holders. 25\% will be reinvested back into the coins for more staking power. 10\% will got to operational costs, server upgrades, etc

\begin{figure}[h]
\centering
\caption{Dividends distribution}
\begin{tikzpicture}[nodes = {font=\sffamily}]
	\foreach \percent/\name in {
		65/Token holders,
		25/New investments,
		10/Costs coverage,
	} {
		  \ifx\percent\empty\else               % If \percent is empty, do nothing
		  \global\advance\cyclecount by 1     % Advance cyclecount
		  \global\advance\ind by 1            % Advance list index
		  \ifnum3<\cyclecount                 % If cyclecount is larger than list
		  \global\cyclecount=0              %   reset cyclecount and
		  \global\ind=0                     %   reset list index
		  \fi
		  \pgfmathparse{\cyclelist[\the\ind]} % Get color from cycle list
		  \edef\color{\pgfmathresult}         %   and store as \color
		    % Draw angle and set labels
		  \draw[fill={\color!50},draw={\color}] (0,0) -- (\angle:\radius)
		  arc (\angle:\angle+\percent*3.6:\radius) -- cycle;
		  \node at (\angle+0.5*\percent*3.6:0.7*\radius) {\percent\,\%};
		  \node[pin=\angle+0.5*\percent*3.6:\name]
		  at (\angle+0.5*\percent*3.6:\radius) {};
		  \pgfmathparse{\angle+\percent*3.6}  % Advance angle
		  \xdef\angle{\pgfmathresult}         %   and store in \angle
		  \fi
	  };
\end{tikzpicture}
\end{figure}

\section{Roadmap}

\begin{vtimeline}[line offset=2pt]
	Early Aug. 2017 & Whitepaper finalized\endlr
	Aug. 2017 & Announcement\endlr
	Sep. 7, 2017 & pre-ICO launch with 10\% bonus\endlr
	Early Sep. 2017 & move servers to new location\endlr
	Sep. 7-21, 2017 & move funds to altcoins to start staking\endlr
	Oct. 1, 2017 & process payment report for pre-ico token holders\endlr
	Oct. 10, 2017 & launch phase 1 ICO 5\% bonus\endlr
	Oct. 2017 & launch BOSCoin masternode\endlr
	Nov. 1, 2017 & process payments for all current token holders\endlr
	Nov. 2017 & POOL token to exchanges\endlr
	Dec. 1, 2017 & process payments for token holders\endlr
	2018 & phase 2 ICO\endlr
	...  & reward payments continue monthly\endlr
\end{vtimeline}


\section{Contact}

Website: \url{http://stakepool.co}

Twitter: \url{https://twitter.com/POSMiningCo}

BitcoinTalk: \url{https://bitcointalk.org/index.php?topic=2105630}

Facebook: \url{https://www.facebook.com/POS-Mining-Co-StakePoolco-136398393631970/}


\bib
\end{document}
